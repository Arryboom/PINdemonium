\chapter{Implementation details}
\label{chapter4}
\thispagestyle{empty}

\section{System architecture}
Our tool is entirely based on PIN, a binary instrumentation framework developed by PIN. It lets us to have the instruction-level granularity useful to insert callback before and after instructions. In this way we are able, for example, to see where a write instruction is going to write and consequently create the write sets.\\
We have integratd Scylla, an external plug-in, to dump the code and reconstruct the IAT. However, we have modified it in order to deal with \textit{IAT Redirection} and \textit{Stolen API} techniques.\\
Finally we use the IDA Pro disassembler and an IDAPython script in the \textit{Init function calls} heuristic. The script calls IDA which reads the imports of the dump and compare them to a list of functions commonly used by the malware and not by the packer (regitry manipulation, internet communication).

\section{System details}