\chapter{Motivation}
\label{chapter2}
\thispagestyle{empty}

\section{Problem statement}
\paragraph{}
In order to avoid Anti-Virus (AV) detection and harden the process of reverse engineering usually malware hide their code employing different techniques. This process, called \textit{packing}, makes the static analysis of a binary completely useless.\\
A \textit{packer} is the tool that implements the previously described functions: it receives in input a binary, transforms and obfuscates its code/resources and then appends new codes that will \textit{unpack} the original one runtime.\\
The complexity of \textit{packers} can be very different: from those which write and execute directly the original code, to others that employ multiple unpacking routines and obfuscation techniques such as runtime repacking of previously unpacked code.\\
This process has different consequences both in the AV detection and manual analysis of malicious binary:
\begin{itemize}
\item Packing a binary destroys any possibility to understand what the program will do on the system without executing it. This problem, with the fact that packing a binary with different packers lead to different signature of the malware, voids any static analysis techniques implemented by AV. 
\item The process of reverse engineering a packed malware can be very time consuming and since lots of malware is pushed every day on internet there is the necessity of fast analysis and fast updating of AV software.
\end{itemize}
\paragraph{}
These problems inspired different works in building an automatic generic unpacker aimed to extract the original code from the packed one. Some of them are more oriented in detection of malicious packed program helping an AV software on end users PCs, others are instead proposed as tools for speed up the work of a professional malware analyst.

\section{State of the art}
\paragraph{}
Lots of different tools using different approaches and techniques have been proposed. 
The approaches for automatic unpacking can be very different:
\begin{itemize}
\item Static unpacking: this can sound counterintuitive, but some works proposed to identify unpacking routines inside the binary and reconstruct an ad hoc unpacker for the binary starting from these routines.
\item Hybrid unpacking: this mixes some static heuristics with dynamic analysis.
\item Dynamic unpacking: this lets the unpacker do its work and tries to extract runtime the unpacked code. 
\end{itemize}
\paragraph{}
Depending on the purpose of the tool there are different requirements that a generic unpacker must respect. If the aim is to help the AV on end users' PCs:
\begin{itemize}
\item Safety: try to recognize the malicious behaviour as fast as possible and block the execution. 
\item Performance: it should not slow down too much the execution of AV scans.
\end{itemize}
Note that in this case, the scope is not to reconstruct a binary from a packed one, but rather to stop malicious behaviours when they manifest. \\
In this area different works have been done, some of them are: OmniUnpack, JustIn.
\paragraph{}
On the other side if the aim is to help the analysis in a lab:
\begin{itemize}
\item Fidelity: the unpacked binary extracted by the tool should be equal to the one that would be unpacked normally. 
\item Generality: the unpacker can not be focused only against one packer but should unpack different of them with one generic algorithm.
\end{itemize}
In this case we do not care so much about safety because usually analysis is performed inside a controlled lab and the analysts want to observe the complete execution of malware. Also the performances are not a critical feature here because we are not constricted by user experience needs.\\
In this category have been developed tools like PolyUnpack, Ether, Eureka, Renovo, Lynx.
These tools merely collect dumps of the binary while unpacking and they fail to reconstruct a fully runnable binary given a packed one.


\section{Goals and challenges}
\paragraph{}
Since our work is born as a component of a bigger malware analysis platform (Jackdaw), our tool is oriented to help malware analyst. 
Our approach aims not only to unpack the malware, but also to reconstruct a fully working unpacked binary. To do so, we not only have to identify the original entry point (OEP) and dump the code at that moment, but we have to find the IAT inside the process and reconstruct a correct import directory in the final PE file.\\
The first thing we have to deal with are the unpacking routines of the packers: every time the execution of the malware comes from a previously written memory area, then it could be a sign that the unpacking stage has finished or that a new unpacking layer has started.\\
We have also to deal with techniques of IAT obfuscation: some malwares can do this in order to make difficult to statically analyse them to understand what they are doing.

