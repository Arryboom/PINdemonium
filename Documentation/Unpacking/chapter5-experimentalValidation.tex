\chapter{Experimental validation}
\label{chapter5}
\thispagestyle{empty}

\section{Goals}
With our experiments we want to show the effectiveness of our tool. During the development we did some preliminary tests on normal programs: we packed them with common packers and tried our tool on them. Then results were pretty convincing, our tool correctly unpack sample programs packed with the following packers:
\begin{itemize}
\item UPX
\item FSG
\item Yoda Crypter
\item mew
\item mpress
\item PECompact
\item ASProtect
\end{itemize}
Moreover, we are able to dump the program at the entry point, but not to reconstruct the IAT, for executables packed with:
\begin{itemize}
\item Themida, but a version without the anti-evasion flag activated
\item Obsidium, but a version without the anti-debugging flag activated
\end{itemize}
We are able to produce not working dumps also for executables packed with ASPack.

\section{Dataset}
We built our dataset from the database of VirusTotal. Using a python script we were able to write a specific query to download only packed executables. We put these malwares in a shared folder between the host and the guest system.

\section{Experimental setup}
In order to automatize our system we used VBoxManage, the command line tool of VirtualBox. We created a .bat script on the host machine that was able to restore, start, wait for 10 minutes and close the VirtualMachine using VBoxManage commands.\\
At the start up of the Virtual Machine, a python script was set to automatically move a malware sample from the shared folder into the Virtual Machine and then trigger the execution of PIN with all the command line arguments to properly analyse the sample. After 5 minutes, it eventually stopped the execution of PIN.\\
The final results were then moved into the shared folder with the host, in order to not lose them at the restoring of the Virtual Machine.

\section{Experiments}

