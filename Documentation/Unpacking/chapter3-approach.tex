\chapter{Approach}
\label{chapter3}
\thispagestyle{empty}

\section{Approach overview}
Our tool has an instruction-level granularity: each instruction is analysed and then goes trough the following steps:
\begin{enumerate}
\item check if it is a write instruction: we track the memory region in which the instruction writes, in order to create a list of memory blocks (we call them "Write sets") of contiguous writes.
\item check if instruction executes from one of the write sets, including write instructions. If this is the case we do the following things:
	\begin{enumerate}
	\item dump the code at this point of execution
	\item reconstruct the IAT of the obtained dump
	\item apply some heuristics to evaluate our dump 
	\end{enumerate}
\item jump to the next instruction
\end{enumerate}
In this way, at the end of the execution of the malware we have a series of dumps and a report which includes the results of each heuristic for every dump. With these information we can choose the best dump, that is the one that has the greatest chance of work.\\

\section{Approach details}
During the development we have adapted our approach in order to increase speed and effectiveness of our tool. Following there is a detailed explanation of our improvements on the initial approach:
\begin{enumerate}
\item in the first step, we add the option of not to track writes of library instructions on the stack and in the teb
\item in the second step we filter instructions of known libraries before dumping
	\begin{enumerate}
	\item in this step there are no improvements
	\item when trying to reconstruct the IAT we added some code in order to deal with 			obfuscation techniques like \textit{IAT Redirection} and \textit{Stolen API}
	\item our heuristics are:
		\begin{itemize}
		\item entropy: check if the value of the entropy is above a certain threshold
		\item long jump: check if the "distance" between the current EIP and the previous 			one is above a certain threshold
		\item jump outer section: check if the current EIP is a different section from the 		one of the previous EIP
		\item pushad popad: check is a pushad popad has been found in the trace
		\item init function calls: check if the imports of the dump are function commonly 			 used by the malware and not by the unpacking layers
		\end{itemize}
	\end{enumerate}
\end{enumerate} 
For the instructions that execute from the same write set we adopted the following approach: if the "distance" between the current EIP and the EIP of the previous instruction is above a given threshold then we do the same as if we were in the case 2, otherwise we jump to the next instruction.\\
Finally, we are noticed that dumping only the main executable in memory is not enough because some packers dump the final payload on the heap. In order to deal with it, we track heap allocations and writes inside an heap interval. If necessary, we dump these intervals too.
